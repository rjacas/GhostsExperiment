\documentclass[11pt,spanish,letterpaper]{report}
\usepackage[right=2cm,left=3cm,top=2cm,bottom=2cm,headsep=0cm,footskip=1cm]{geometry}
\usepackage[T1]{fontenc}
\usepackage[utf8]{inputenc}
\usepackage[spanish]{babel}
\usepackage{graphicx}
\title{Instrucciones de usabilidad de Ghosts}
\author{Ricardo Jacas}

\begin{document}
\maketitle
%\vspace{-1cm}

\section*{Instrucciones generales}

A continuación se presentan dos problemas que usted deberá resolver utilizando \textit{Test Driven Development}. Para ello, antes de comenzar el desarrollo, deberá completar los casos de prueba que se indican, en cada problema.

Para cada problema cree un nuevo proyecto Java.

Para crear sus archivos de \textit{Testing} utilice el asistente de creación de Eclipse y seleccione JUnit 4 como opción de biblioteca. Además agregue en el comienzo de sus archivos la instrucción \verb|import junit.framework.TestCase;| y extienda de \textit{TestCase}. Esto le permitirá agregar el método \verb|setUp()| para inicializar sus variables antes de utilizarlas en cada \textit{test}. Recuerde anotar cada método de \textit{test} con la anotación \verb|@Test|, esto le permitirá utilizar el asistente de Eclipse para correr sus \textit{tests}.    

\end{document}